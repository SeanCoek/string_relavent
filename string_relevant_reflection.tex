\documentclass{article}
% \usepackage[UTF8]{ctex}
\usepackage{paralist}
\usepackage{geometry}
\usepackage[colorlinks, linkcolor=black]{hyperref}
\newcommand{\tabincell}[2]{\begin{tabular}{@{}#1@{}}#2\end{tabular}}
\geometry{a4paper, scale=0.65}
\author{Xilong Zhuo}
\title{String Relevant Reflection APIs}
\begin{document}
	\maketitle

	\section{Reflection}
	In this section we introduce the reflection APIs in Java and list all the APIs related to string.\\
	Reflection can enable programs to examine or modify the runtime behavior of applications running in the Java virtual machine. Reflection APIs can be divided into 4 categories which are represented below respectively as \emph{obtain a class},\emph{create a object},\emph{access to field} and \emph{access to method}. String relevant APIs will be mark with "*" at the end of its name. Static methods showed in table below will be written as in italic.	
	\subsection{Obtain a class}
	\emph{java.lang.Class} is the entry point for all reflection operations. 
	Table 1 shows several ways to get a \emph{Class}.
	\begin{table}[hp]
	\begin{tabular}{p{7.5cm}p{7cm}}
	\hline
	\textbf{API}&\textbf{Description}\\
	\hline
	java.lang.Object.getClass()&only works for reference types which all inherit from \emph{Object}\\
	\hline
	the \textbf{.class} Syntax&Confused here, At first it says the code in this site is correct, but then indicate that statement would produce a compile-time error. \href{https://docs.oracle.com/javase/tutorial/reflect/class/classNew.html}{Here}\\
	\hline
	\emph{java.lang.Class.forName(String className)*}&fully-qualified name of a class is available.The syntax for names of array classes is described by \href{https://docs.oracle.com/javase/8/docs/api/java/lang/Class.html#getName--}{Class.getName()}\\
	\hline
	\emph{java.lang.Class.forName(String name, boolean initialize, ClassLoader loader)*}& - \\
	\hline
	the \textbf{TYPE} field&for primitive types and \emph{void}\\
	\hline
	\tabincell{l}{java.lang.Class.getSuperclass()\\java.lang.Class.getClasses()\\java.lang.Class.getDeclaredClasses()\\java.lang.Class.getDelaringClass()\\java.lang.reflect.Field.getDeclaringClass()\\java.lang.reflect.Method.getDeclaringClass()\\java.lang.reflect.Constructor.getDeclaringClass()}&these methods can only be accessed if a class has already been obtained\\
	\hline
	\end{tabular}
	\caption{several ways to get a class}
	\end{table}

	\subsection{Create a object}
	Reflection provides tow ways to create a object. They behaves very much like the \emph{new} keyword. The details is showed in Table 2.
	\begin{table}[hp]
	\begin{tabular}{p{7.5cm}p{7cm}}
	\hline
	\textbf{API}&\textbf{Description}\\
	\hline
	\emph{java.lang.Class.newInstance()}&It could only call the empty-parameter constructor and fails when that constructor can't be access(non-existent or private). But this is the commonest usage.\\ 
	\hline
	java.lang.reflect.Constructor.newInstance(Object... initargs)*&\\
	\hline
	\end{tabular}
	\caption{APIs to create a object}
	\end{table}
	The constructor object can be obtained by invoking \emph{java.lang.Class.getConstructor(Class$<?>$... parameterTypes)} or \emph{java.lang.Class.getConstructors()}.

	\subsection{Access to field}
	The \emph{java.lang.reflect.Field} class provides methods for accessing type information and setting and getting values of a field on a given object. Details in table 3.
	\begin{table}[hp]
	\begin{tabular}{p{8cm}p{7cm}}
	\hline
	\textbf{API}&\textbf{Description}\\
	\hline
	\tabincell{l}{java.lang.Class.getField(String name)*\\java.lang.Class.getFields()\\java.lang.Class.getDeclaredFields()\\java.lang.Class.getDeclaredField(String name)*}&APIs with name including "Declared" can return not only public field.\\
	\hline
	java.lang.reflect.Field.getModifiers()&\\
	\hline
	java.lang.reflect.Field.getName()&\\
	\hline
	\tabincell{l}{java.lang.reflect.Field.set(Object obj,\\ Object value)*\\java.lang.reflect.Field.setBoolean(Object obj,\\ boolean z)\\setLong, setByte,...}&the first parameter represents a certain object which these operation perform on.\\
	\hline
	\tabincell{l}{java.lang.reflect.Field.get(Object obj)\\java.lang.reflect.Field.get...}& get a value from certain object at runtime\\
	\hline

	\end{tabular}
	\caption{Access to method}
	\end{table}

	\subsection{Access to method}
	The \emph{java.lang.reflect.Method} class provides APIs to access information about a method's modifiers, return type, parameters, annotations, and thrown exceptions. It also be used to invoke methods.
	\begin{table}[hp]
	\begin{tabular}{p{7.5cm}p{7cm}}
	\hline
	\textbf{API}&\textbf{Description}\\
	\hline
	\tabincell{l}{java.lang.Class.getMethod(String name,\\ Class$<?>$... parameterTypes)\\java.lang.Class.getMethods()\\java.lang.Class.getDeclaredMethod(String name,\\ Class$<?>$... parameterTypes)\\java.lang.Class.getDeclaredMethods()}&same as field\\
	\hline
	java.lang.reflect.Method.getModifiers()&\\
	\hline
	java.lang.reflect.Method.getName()&\\
	\hline
	java.lang.reflect.Method.getParameters()&\\
	\hline
	\tabincell{l}{java.lang.reflect.Method.invoke(Object obj,\\ Object... args)*}& invoking a method at runtime.\\
	\hline
	\end{tabular}
	\caption{access to method}
	\end{table}

\end{document}